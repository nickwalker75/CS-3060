\documentclass[paper=letter, fontsize=11pt]{scrartcl} % A4 paper and 11pt font size

\usepackage{enumitem}
\usepackage{listings,multicol}
\usepackage[T1]{fontenc} % Use 8-bit encoding that has 256 glyphs
\usepackage{fourier} % Use the Adobe Utopia font for the document - comment this line to return to the LaTeX default
\usepackage[english]{babel} % English language/hyphenation
\usepackage{amsmath,amsfonts,amsthm} % Math packages
\usepackage{lipsum} % Used for inserting dummy 'Lorem ipsum' text into the template
\usepackage{sectsty} % Allows customizing section commands
\allsectionsfont{\centering \normalfont\scshape} % Make all sections centered, the default font and small caps
\usepackage{fancyhdr} % Custom headers and footers
\pagestyle{fancyplain} % Makes all pages in the document conform to the custom headers and footers
\fancyhead{} % No page header - if you want one, create it in the same way as the footers below
\fancyfoot[L]{} % Empty left footer
\fancyfoot[C]{} % Empty center footer
% \fancyfoot[R]{\thepage} % Page numbering for right footer
\renewcommand{\headrulewidth}{0pt} % Remove header underlines
\renewcommand{\footrulewidth}{0pt} % Remove footer underlines
\setlength{\headheight}{13.6pt} % Customize the height of the header

\setlength\parindent{0pt} % Removes all indentation from paragraphs - comment this line for an assignment with lots of text

\usepackage[margin=0.75in]{geometry}
\usepackage{hyperref}
%----------------------------------------------------------------------------------------
%   TITLE SECTION
%----------------------------------------------------------------------------------------

\newcommand{\horrule}[1]{\rule{\linewidth}{#1}} % Create horizontal rule command with 1 argument of height

\title{ 
    \normalfont \normalsize 
    \textsc{CS 3060 Programming Languages, Spring 2021} \\ [25pt] % Your university, school and/or department name(s)
    \horrule{0.5pt} \\[0.4cm] % Thin top horizontal rule
    \huge Assignment \#9  \\ % The assignment title
    \horrule{2pt} \\[0.5cm] % Thick bottom horizontal rule
}

% \author{John Smith} % Your name

% \date{\normalsize\today} % Today's date or a custom date

\begin{document}

    \begin{center}
         Assignment \#9\\
        \small CS 3060 Programming Languages, Spring 2021 \\
        \small Instructor: Dr. Roy \\
        \huge Scala-Haskell 
    \end{center}
    
    \textbf{Due Date:} Apr 24 (Sat) @ 11.59 PM. \\
    \textbf{Total Points:} 50 points \\
    \textbf{Note:} This is an extra assignment, which may help you improve your grade as we will be picking up your best 8 assignments out of 9 assignments.\\

    \textbf{Directions:} Using the source provided via Gitlab \@ \texttt{https://gitlab.com/sanroy/sp21-cs3060-hw/},
complete the assignment below. The process for completing this assignment should be as follows:

    \begin{enumerate}[noitemsep]
        \item You already forked the Repository ``sanroy/sp21-cs3060-hw'' to a repository ``yourId/sp21-cs3060-hw'' under your username. If not, do it now.
        \item Get a copy of hw9 folder in ``sanroy/sp21-cs3060-hw'' repository as a hw9 folder in your repository ``yourId/sp21-cs3060-hw''
        \item Complete the assignment, committing changes to git. Each task code should be in a separate file. As an example, task1.scala for Task 1.
        \item Push all commits to your Gitlab repository
        \item If you have done yet done so, sdd TA (username: prabeshpaudel) as a member of your Gitlab repository
    \end{enumerate}


    \textbf{Tasks:}
    \begin{enumerate}

       \item \textbf{Task \#1: (10 points)} Scala. In Task 6 of HW5 (Scala) we wrote a Scala program to process 
two textfiles (i.e. two stories) that you downloaded from a website (http://www.textfiles.com/stories/). The 
Scala program was to compute some statistics of those two files (e.g. word count, etc.). 
I guess you will find it even more exciting if we can automatically grab all story files 
(maybe, more than 100 stories it has) that appear on the same website
and process them to collect statistics. That is exactly what you will do in the current task.
In particular, write a Scala program to process all stories (whereas each story is a text file) present on
 http://www.textfiles.com/stories/ and report the average number of words in a story. 

Hint: In HW6 we learned how to programmatically find all links on a webpage. Here each link corresponds to one story textfile.   
 
       \item \textbf{Task \#2: (6 points)} Scala. Let's get back to Task 2 of HW6 (Scala) where we developed a mini web crawler. 
You know that in Task 2c we did not use parallel computation (i.e. multi-Threaded processing) last time. 
Your current task is to do exactly that, i.e., redo Task 2c through parallel computation.  
You need to use the concept of \texttt{par} collection, and concept of functional programming while ensuring that your program is free from \texttt{side effect}. 
\emph{Writing README carries 1 point.}


        \item \textbf{Task \#3a: (10 points)} Haskell. Write a Haskell function named \texttt{myCount} 
which takes two parameters, an item \texttt{p} and a \texttt{list}. 
The myCount function returns how many times \texttt{p} appears in \texttt{list}. 
Your function needs to be generic, e.g. should work for integer, characters or anything else. 
Test your function as follows and report the result in README.

myCount 'a' ['a','b','a','b','a'] should return 3

myCount  3 [5,1,5,2,3]  should return 1

\textbf{Task \#3b: (3 points)} 
Test your function on a list of Cards. The Card type is defined in Haskell lecture ppt 2.  
Report the result in README.

\textbf{Task \#3c: (3 points)} 
Test your function on a list of Books. The Book type is defined in Haskell group activity 15.  
Report the result in README.

\emph{Writing README carries 2 points.}
        \item \textbf{Task \#4: (18 points)} Haskell. Haskell recursive type: 
Write a function \texttt{greaterThan10} which takes a binary tree as input and returns 
the number of nodes (in the tree) that hold value greater than 10. 
Use the following data type for the binary tree, 
where each internal node as well as each leaf node holds a value.

data Tree a = Leaf a | BinaryTree (Tree a) a (Tree a)  deriving (Show)

Build a tree of 12 or more nodes with height 3 or more 
(where each node holding an integer) and apply your \texttt{greaterThan10} 
function on the tree, and report the output in README file. \emph{Writing README carries 2 points.}

    % \vspace{2cm}
    
    \end{enumerate}

\end{document}


