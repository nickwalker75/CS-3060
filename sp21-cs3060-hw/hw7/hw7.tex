\documentclass[paper=letter, fontsize=11pt]{scrartcl} % A4 paper and 11pt font size

\usepackage{enumitem}
\usepackage{listings,multicol}
\usepackage[T1]{fontenc} % Use 8-bit encoding that has 256 glyphs
\usepackage{fourier} % Use the Adobe Utopia font for the document - comment this line to return to the LaTeX default
\usepackage[english]{babel} % English language/hyphenation
\usepackage{amsmath,amsfonts,amsthm} % Math packages
\usepackage{lipsum} % Used for inserting dummy 'Lorem ipsum' text into the template
\usepackage{sectsty} % Allows customizing section commands
\allsectionsfont{\centering \normalfont\scshape} % Make all sections centered, the default font and small caps
\usepackage{fancyhdr} % Custom headers and footers
\pagestyle{fancyplain} % Makes all pages in the document conform to the custom headers and footers
\fancyhead{} % No page header - if you want one, create it in the same way as the footers below
\fancyfoot[L]{} % Empty left footer
\fancyfoot[C]{} % Empty center footer
% \fancyfoot[R]{\thepage} % Page numbering for right footer
\renewcommand{\headrulewidth}{0pt} % Remove header underlines
\renewcommand{\footrulewidth}{0pt} % Remove footer underlines
\setlength{\headheight}{13.6pt} % Customize the height of the header

\setlength\parindent{0pt} % Removes all indentation from paragraphs - comment this line for an assignment with lots of text

\usepackage[margin=0.75in]{geometry}
\usepackage{hyperref}
%----------------------------------------------------------------------------------------
%   TITLE SECTION
%----------------------------------------------------------------------------------------

\newcommand{\horrule}[1]{\rule{\linewidth}{#1}} % Create horizontal rule command with 1 argument of height

\title{ 
    \normalfont \normalsize 
    \textsc{CS 3060 Programming Languages, Spring 2021} \\ [25pt] % Your university, school and/or department name(s)
    \horrule{0.5pt} \\[0.4cm] % Thin top horizontal rule
    \huge Assignment \#7  \\ % The assignment title
    \horrule{2pt} \\[0.5cm] % Thick bottom horizontal rule
}

% \author{John Smith} % Your name

% \date{\normalsize\today} % Today's date or a custom date

\begin{document}

    \begin{center}
         Assignment \#7\\
        \small CS 3060 Programming Languages, Spring 2021 \\
        \small Instructor: S. Roy \\
        \huge Haskell \#1
    \end{center}
    
    \textbf{Due Date:} Apr 13 @ 11.59 PM. \\
    \textbf{Total Points:} 50 points \\

    \textbf{Directions:} Using the source provided via Gitlab \@ \texttt{https://gitlab.com/sanroy/sp21-cs3060-hw/},
complete the assignment below. The process for completing this assignment should be as follows:

    \begin{enumerate}[noitemsep]
        \item You already forked the Repository ``sanroy/sp21-cs3060-hw'' to a repository ``yourId/sp21-cs3060-hw'' under your username. If not, do it now.
        \item Get a copy of hw7 folder in ``sanroy/sp21-cs3060-hw'' repository as a hw7 folder in your repository ``yourId/sp21-cs3060-hw''
        \item Complete the assignment, committing changes to git. Each task code should be in a separate file. As an example, task1.hs for Task 1.
        \item Push all commits to your Gitlab repository
        \item If you have done yet done so, sdd TA (username: prabeshpaudel for CS 3060) as a member of your Gitlab repository
    \end{enumerate}


    \textbf{Tasks:}
    \begin{enumerate}

        \item \textbf{Task \#1: (12 points)} Write a Haskell program that prints the string ``Hello, NAME, you are hard working.'' where NAME is your name. 
Note that you need to compile your .hs file to create an executable, using commands like, ``ghc -o myhello  prog.hs'' and then run the ``myhello" executable.
Submit a screenshot that shows the above activities (which carries 4 points). \emph{Writing README carries 1 point.}

        \item \textbf{Task \#2: (12 points)} Write a function that accepts a list (lst) of integers as the parameter, and 
filters out a sub-list (of lst) which contains only the 3's multiples and 5's multiples in lst (if any). 
As an example, if lst is [34,2,12,25,15,32, 20], then the output is [12,25,15,20]. \emph{Writing README carries 1 point.}

        \item \textbf{Task \#3: (14 points)} Write a function that accepts a list (lst) of integers as the parameter, and 
returns $x$ where $x$ is the number of perfect squares in the list lst. \emph{Writing README carries 1 point.}
As an example, if lst is [34,9,80,16,225,15,1000], then the output is 3 as there are three perfect squares (9, 16, and 225).
        
	\item \textbf{Task \#4: (12 points)} Write a Haskell function (named toDigitList) that accepts an integer 
as the parameter, and if the integer is a non-negative integer then it breaks it into its digits and the output is a list of digits. 
On the other hand, if the input is a negative integer, the function returns an empty list. 
Use the following examples to test your function: 

(a) toDigitList 1325 gives output [1,3,2,5]
(b) toDigitList 0 gives output [0]
(c) toDigitList (-32) gives output []

\emph{Writing README carries 2 points.}

    % \vspace{2cm}
    
    \end{enumerate}

\end{document}


