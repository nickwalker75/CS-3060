\documentclass[paper=letter, fontsize=11pt]{scrartcl} % A4 paper and 11pt font size

\usepackage{enumitem}
\usepackage{listings,multicol}
\usepackage[T1]{fontenc} % Use 8-bit encoding that has 256 glyphs
\usepackage{fourier} % Use the Adobe Utopia font for the document - comment this line to return to the LaTeX default
\usepackage[english]{babel} % English language/hyphenation
\usepackage{amsmath,amsfonts,amsthm} % Math packages
\usepackage{lipsum} % Used for inserting dummy 'Lorem ipsum' text into the template
\usepackage{sectsty} % Allows customizing section commands
\allsectionsfont{\centering \normalfont\scshape} % Make all sections centered, the default font and small caps
\usepackage{fancyhdr} % Custom headers and footers
\pagestyle{fancyplain} % Makes all pages in the document conform to the custom headers and footers
\fancyhead{} % No page header - if you want one, create it in the same way as the footers below
\fancyfoot[L]{} % Empty left footer
\fancyfoot[C]{} % Empty center footer
% \fancyfoot[R]{\thepage} % Page numbering for right footer
\renewcommand{\headrulewidth}{0pt} % Remove header underlines
\renewcommand{\footrulewidth}{0pt} % Remove footer underlines
\setlength{\headheight}{13.6pt} % Customize the height of the header

\setlength\parindent{0pt} % Removes all indentation from paragraphs - comment this line for an assignment with lots of text

\usepackage[margin=0.75in]{geometry}
\usepackage{hyperref}
%----------------------------------------------------------------------------------------
%   TITLE SECTION
%----------------------------------------------------------------------------------------

\newcommand{\horrule}[1]{\rule{\linewidth}{#1}} % Create horizontal rule command with 1 argument of height

\title{ 
    \normalfont \normalsize 
    \textsc{CS 3060 Programming Languages, Spring 2021} \\ [25pt] % Your university, school and/or department name(s)
    \horrule{0.5pt} \\[0.4cm] % Thin top horizontal rule
    \huge Assignment \#8  \\ % The assignment title
    \horrule{2pt} \\[0.5cm] % Thick bottom horizontal rule
}

% \author{John Smith} % Your name

% \date{\normalsize\today} % Today's date or a custom date

\begin{document}

    \begin{center}
         Assignment \#8\\
        \small CS 3060 Programming Languages, Spring 2021 \\
        \small Instructor: S. Roy \\
        \huge Haskell \#2
    \end{center}
    
    \textbf{Due Date:} Apr 20 @ 11:59 PM. \\
    \textbf{Total Points:} 50 points \\

    \textbf{Directions:} Using the source provided via Gitlab \@ \texttt{https://gitlab.com/sanroy/sp21-cs3060-hw/},
complete the assignment below. The process for completing this assignment should be as follows:

    \begin{enumerate}[noitemsep]
        \item You already forked the Repository ``sanroy/sp21-cs3060-hw'' to a repository ``yourId/sp21-cs3060-hw'' under your username. If not, do it now.
        \item Get a copy of hw8 folder in ``sanroy/sp21-cs3060-hw'' repository as a hw8 folder in your repository ``yourId/sp21-cs3060-hw''
        \item Complete the assignment, committing changes to git. Each task code should be in a separate file. As an example, task1.hs for Task 1.
        \item Push all commits to your Gitlab repository
        \item If you have done yet done so, sdd TA (username: prabeshpaudel) as a member of your Gitlab repository
    \end{enumerate}




    \textbf{Tasks:}
    \begin{enumerate}

        \item \textbf{Task \#1: (30 points)} Write a Haskell function for each of the following. In your code, you need to specify the \textbf{input and output type} of each function. 

(a) (10 points) Write a Haskell function \emph{bar} which takes an integer \emph{x} as input parameter 
and returns the sum of all positive odd integer(\emph{y})s' cubes whereas \emph{y} is smaller than \emph{x}. 
You need to use \emph{foldl} to do the above computation. 
As an example, if \emph{x} is 10, then \emph{bar} will compute (${1^3} + {3^3} + \ldots + {9^3}$).
\emph{Writing README carries 1 point.}


(b)(10 points) Write a Haskell function \emph{charCount} which takes a string \emph{word} as input, 
and counts how many letters (say $c1$) in \emph{word} are in uppercase and how many letters 
(say $c2$) in \emph{word} are in lowercase, and returns two counts ($c1$, $c2$) as a single tuple. 
As an example, if \emph{word} is "abDfGi", then \emph{charCount} returns (2,4). 
\emph{Writing README carries 1 point.}

(c)(10 points) Write a Haskell function \emph{longStrCount} which takes a list of strings as input, and 
counts how many strings have length more than 7, and returns the count. 
As an example, if input \emph{list} is ["abcd", "de", "fghtestwsd"], then \emph{longStrCount} returns 1. 
\emph{Writing README carries 1 point.}
 
       \item \textbf{Task \#2: (20 points)} Refer to the user-defined types \texttt{Card} and \texttt{Hand} in the textbook (cards-with-show.hs).
Also, see the \texttt{value} and \texttt{cardvalue} function therein. Write a Haskell function for each of the following. 
In your code, you need to specify the \textbf{input and output type} of each function. 

(a) (7 points) Write a function named \texttt{lowerCard} which takes two Cards and 
returns the lower value Card. If there is a tie, then either Card can be returned.

(b) (6 points) Write a function named \texttt{productValue} which takes a Hand and 
returns the product of all values of cards in that Hand.

(c) (7 points) Write a function named \texttt{higherHand} which takes two Hands and 
returns the Hand which has the higher \texttt{productValue}. If there is a tie, then either Hand can be returned.

\emph{Writing README carries 2 points.}

    % \vspace{2cm}
    
    \end{enumerate}

\end{document}


